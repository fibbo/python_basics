\documentclass[10pt, a4paper]{beamer} %handout fuer eine gedruckte version der slides
\usetheme{metropolis}

\usepackage{url}
\usepackage{polyglossia}
\setmainlanguage{english}

\usepackage{listings,xcolor}
\usepackage{graphicx}
\usepackage{nameref}
\usepackage{mathtools}

\usepackage{booktabs}

\makeatletter
\newcommand*{\currentname}{\@currentlabelname}
\makeatother

\usepackage{fontspec}
% \setmainfont{UbuntuMono}
\newfontfamily\Bera{Bitstream Vera Sans Mono}[Scale=0.85]

\definecolor{mDarkTeal}{HTML}{23373b}
\definecolor{mLightBrown}{HTML}{EB811B}
\definecolor{mDarkBlue}{HTML}{0F3470}
\definecolor{mDarkGrey}{HTML}{999999}
\definecolor{mLighterGrey}{HTML}{CCCCCC}
\definecolor{mLightGrey}{rgb}{0.95,0.95,0.95}
\definecolor{mLightBlue}{HTML}{212751}

\newcommand{\lb}[1]{{\color{mLightBrown}#1}}

\newcommand{\bblock}[3]{
        \begin{block}{#1}
        #2
    \end{block}
}

\setbeamercolor{block title}{fg=mDarkTeal}

\lstset{language=Python,
    backgroundcolor=\color{mLightGrey},
    frame=single,
    rulecolor=\color{mLightGrey}, % make frame "invisible"
    basicstyle=\Bera\footnotesize,
    keywordstyle=\bfseries\color{mDarkBlue},
    commentstyle=\color{mDarkGrey}\itshape,
    captionpos=t,
    emphstyle=\ttb\color{mDarkBlue},    % Custom highlighting style
    stringstyle=\color{mLightBrown},
    tabsize=2,
    numberbychapter=false,
    showstringspaces=false,
    breaklines=true,
    morekeywords={__init__, and, assert, break, class, continue, def, del, elif, else, except, exec, finally, for, from, global, if, import, in, is, lambda, not, or, pass, print, raise, return, try, while, yield}
}
%


\setbeamertemplate{section in toc}{%
  {\color{mDarkTeal}\rule[0.3ex]{3pt}{3pt}}~\inserttocsection\par}
\setbeamertemplate{subsection in toc}{%
  \hspace{1.2em}{\color{mLightBrown}\rule[0.3ex]{3pt}{3pt}}~\inserttocsubsection\par}


% \usecolortheme{spruce}
\AtBeginSection[]
{
  \begin{frame}
    \frametitle{\currentname}
    \tableofcontents[currentsection]
  \end{frame}
}

\AtBeginSubsection[]
{
  {
  % \setbeamercolor{background canvas}{bg=mLighterGrey}
  \begin{frame}
    \frametitle{\currentname}
    \tableofcontents[currentsection]
  \end{frame}
  }
}

\title % (optional, only for long titles)
{Foundations of Programming in Python}
\author % (optional, for multiple authors)
{Philipp Gloor\inst{1}}
\institute
{
  \inst{1}%
  University of Zurich
}
\date{}
\subject{Python}
\titlegraphic{\includegraphics[width=0.3\textwidth]{pics/uzh_logo_e_pos}}
\begin{document}
\begin{frame}
\titlepage
\end{frame}

\begin{frame}
\frametitle{About me}

\begin{block}{Education}
    \begin{itemize}
        \item 2012 -- Bachelor of Science UZH in Physics
        \item 2016 -- Master of Science UZH in Computational Science
    \end{itemize}
\end{block}

\begin{block}{Work}
    \begin{itemize}
        \item 2014 -- 2016 Software engineer CERN (remote)
        \item 2016 -- now PDF Tools AG
    \end{itemize}
\end{block}

\begin{block}{Programming experience}
    \begin{itemize}
        \item[] C++, C\#, Java, JavaScript, Python
    \end{itemize}
\end{block}

\begin{block}{Email}
\begin{itemize}
    \item[] philipp.gloor@gmail.com
\end{itemize}
    
\end{block}
 
% In this slide, some important text will be
% \alert{highlighted} beause it's important.
% Please, don't abuse it.
 
% \begin{block}{Remark}
% Sample text
% \end{block}
 
% \begin{alertblock}{Important theorem}
% Sample text in red box
% \end{alertblock}
 
% \begin{examples}
% Sample text in green box. "Examples" is fixed as block title.
% \end{examples}
\end{frame}
\begin{frame}[t]\frametitle{Round of introduction}
    \begin{itemize}
        \item Name
        \item Occupation
        \item Programming experience? What language?
        \item Expectations
    \end{itemize}
\end{frame}

\begin{frame}[t]\frametitle{Learning targets}
    
After this course...
\begin{itemize}
    \item ... you will know what programming is
    \item ... you will know how to write a basic computer program
    \item ... you will know the fundamental components of programming
    \item ... you are able to run Python code
    \item ... you are able to write a Python program based on a written out
    problem statement
    \item ... you know where you can find more information to improve your
    programming skills
\end{itemize}
\end{frame}


\section{Introduction to Programming} % (fold)
\label{sec:introduction_to_programming}

\begin{frame}[c]\frametitle{What is a Computer Program}
\begin{block}{Modular System}
    \begin{itemize}
        \item \textbf{Input}: Data input from keyboard, files, internet, etc...
        \item \textbf{Output}: Processed data is displayed or saved to a file
        \item \textbf{Algorithms:}
        \begin{itemize}
            \item \textbf{Assignment}: Values are assigned to variables
            \item \textbf{Conditional execution}: Statements are executed only if certain
            conditions are fulfilled
            \item \textbf{Loops}: Repeating statement or group of statements
        \end{itemize}
        
        \item \textbf{Libraries}: Using existing implementations
    \end{itemize}

\end{block}     
\end{frame}

\begin{frame}[fragile,c,allowframebreaks]\frametitle{Examples: Hello World}
    \begin{block}{Java}
    {
    % \lstset
    \begin{lstlisting}[language=Java]
        public class HelloWorld {
            public static void main(String args[]) {
                System.out.println("Hello World");
            }
        }
    \end{lstlisting}    
    }
    \end{block}
    
    \begin{block}{C++}
    {
    % \lstset
    \begin{lstlisting}[language=C++, morekeywords=include]
        #include <iostream>
        int main() {
            std::cout << "Hello World" << std::endl;
            return 0;
        }
    \end{lstlisting}    
    }
    \end{block}
    \framebreak
    \begin{block}{Python}
        \begin{lstlisting}
            print("Hello World")
        \end{lstlisting}
    \end{block}
    
\end{frame}

\begin{frame}[c]\frametitle{Why Python?}
    \begin{columns}
        \begin{column}{0.5\textwidth}
           \begin{itemize}
                \item "Simple" syntax
                \item High-level programming language
                \item Cross-platform
                \item Interpreted
                \item Object-oriented
                \item Many libraries available
            \end{itemize}
        \end{column}
        \begin{column}{0.5\textwidth}  %%<--- here
            \begin{figure}
                \includegraphics[width=0.9\linewidth]{pics/python.png}
            \end{figure}
            \tiny Source: https://xkcd.com/353/
        \end{column}
    \end{columns}

\end{frame}

\begin{frame}[c]\frametitle{Development Environment}

\begin{itemize}
    \item Integrated Development Environment (IDE)
    \item Collection of tools that are commonly used for software development (they make our life easier!)
    \item Popular IDEs
    \begin{itemize}
        \item Eclipse with pydev - \url{http://pydev.org}
        \item JetBrains PyCharm - Community Edition available for free \url{http://jetbrains.com/pycharm/download}
    \end{itemize}
\end{itemize}
\end{frame}

\begin{frame}[fragile,c]\frametitle{Demo: Hello World}

\begin{block}{Options to run Python code:}
\begin{itemize}
    \item Directly in the Python prompt
    \item Write the code into a file and run python with the file
    \item Use IDE to run Python code
\end{itemize}



\end{block}
    
\end{frame}
% section introduction_to_programming (end)

\section{Fundamental Concepts} % (fold)
\label{sec:fundamental_concepts}

\subsection{Values, Variables, Expressions, Operators, Comments} % (fold)
\label{sub:values_variables_expressions_operators_comments}

\begin{frame}[c]\frametitle{Values, Variables, Expressions, Operators, Comments}
\begin{block}{Values}
    \begin{itemize}
        \item Numbers
        \begin{itemize}
            \item 2
            \item 1000000
            \item -2
            \item 3.2
            \item 1.3333333
        \end{itemize}
        \item Strings (Text)
        \begin{itemize}
            \item {\color{blue}'Hello World'}
            \item {\color{red}"Hello World"}
        \end{itemize}
    \end{itemize}
\end{block}
    
\end{frame}

\begin{frame}[c]\frametitle{Data Types}
    \begin{block}{Strings}
    \begin{itemize}
        \item {\color{blue} 'Single quotes'} or {\color{red} "double quotes"} can be used to declare them
        \begin{itemize}
            \item 'Hello World'
            \item "Hello World"
            \item "5"
        \end{itemize}
    \end{itemize}
    \end{block}
    \begin{block}{Boolean}
    Binary data type
        \begin{itemize}
            \item True
            \item False
        \end{itemize}
    \end{block}

\end{frame}

\begin{frame}[c,fragile,allowframebreaks]\frametitle{Variables}
    \begin{itemize}
        \item Variables hold values
        \item Similar to mathematics
        \begin{itemize}
            \item x = 2
            \item y = x + 2
        \end{itemize}
        \item Values assigned using the \texttt{=} operator
    \end{itemize}
    \begin{examples}
    Use meaningful names
    \begin{itemize}
        \item Declaration
        \begin{lstlisting}
            salutation = "Hello"
            name = "Dennis Reynolds"
            pi = 3.13159
        \end{lstlisting}
        \item Usage
        \begin{lstlisting}
            print(name)
        \end{lstlisting}
    \end{itemize}
        
    \end{examples}
\framebreak
\begin{block}{Keywords - reserved words}

\begin{lstlisting}
and, assert, break, class, continue, def, del, elif, else, except, exec, 
finally, for, from, global, if, import, in, is, lambda, not, or, pass, 
print, raise, return, try, while, yield
\end{lstlisting}
    
\end{block}
\framebreak
\begin{block}{Variables and values can be combined}
    \begin{lstlisting}
print(2+2)
a = 2
print(a+2)

salutation = "Hello"
name = "Dennis Reynolds"
print(salutation + " " + name)
    \end{lstlisting}
\end{block}
\end{frame}

\begin{frame}[t, fragile]\frametitle{Operators}
\begin{block}{Order of precedence}
\begin{itemize}
        \item ()
        \item **
        \item unary +,-
        \item *,/,\%
        \item binary +,-
        \item \texttt{<}, \texttt{>}, \texttt{<=}, \texttt{>=}, \texttt{!=}, \texttt{==}
        \item \textbf{\texttt{\color{mDarkBlue}not}}
        \item \textbf{\texttt{\color{mDarkBlue}and}}
        \item \textbf{\texttt{\color{mDarkBlue}or}}
    \end{itemize}    
\end{block}
    
\end{frame}

\begin{frame}[t, fragile]\frametitle{Comments}
\begin{itemize}
    \item Comments have no impact on the program
    \item Should explain the code
    \item A comment starts with a \# character
\end{itemize}

\begin{examples}
    \begin{lstlisting}
# Declaring the name
name = "Philipp"
print(name) # Prints Philipp
    \end{lstlisting}
\end{examples}
    


\end{frame}

% subsection values_variables_expressions_operators_comments (end)

\subsection{Functions} % (fold)
\label{sub:functions}
\begin{frame}[c, fragile,allowframebreaks]\frametitle{Functions}
\begin{itemize}
    \item \lstinline!print()! is a function that you have already used
    \item A function can take arguments which can be used inside the function
    \begin{lstlisting}
name = "Some name"
print(name) # Some name is used inside the print function
    \end{lstlisting}
    \item Functions can also return a result
    \begin{itemize}
        \item \lstinline!return! statement
    \end{itemize}
\end{itemize}

\begin{examples}
    \begin{lstlisting}
text = "Python programming language"
print(text) # Prints: Python programming language
text_length = len(text)
print(text_length) # Prints length of the string
    \end{lstlisting}
\end{examples}
\framebreak
\begin{block}{Type conversions}
\begin{itemize}
    \item \lstinline!int('32')!: Converts a string that holds a number to an integer
    \item \lstinline!int('Hello')!: This doesn't work and it will throw a ValueError exception
    \item \lstinline!float('313.333')!: Converts a string that hold a number to a float
    \item \lstinline!str(32)!: Converts a number to a string
\end{itemize}

\begin{examples}
\begin{lstlisting}
a = 20
b = 10
res = a + b
print("The sum of " + str(a) + " and " + str(b) + " is " + str(res))
\end{lstlisting}
\end{examples}
\end{block}

\framebreak

\begin{block}{Rounding}
    \begin{lstlisting}
a = 1.888
int(a) # = 1
int(round(a)) # = 2
int(a+5) # = 2
    \end{lstlisting}
\end{block}
\begin{block}{Math functions}
    \begin{lstlisting}
import math
log_res = math.log(17.0)
sin_res = math.sin(45)
angle = 20
x = math.cos(20*math.pi/180))  # cos/sin etc take radians as arguments -> conversion from degree to radians necessary      
    \end{lstlisting}
    \begin{itemize}
        \item \tiny \url{http://docs.python.org/library/math.html}
    \end{itemize}
\end{block}

\framebreak

\begin{block}{User-defined functions}
    \begin{itemize}
        \item A function encapsulates some functionality
        \item Reduces complexity
        \begin{lstlisting}
def my_function(param1, param2):
    print(param1)
    print(param2)            
        \end{lstlisting}
        \item Syntax is important
        \begin{itemize}
            \item Indentation
            \item The colon
        \end{itemize}
    \end{itemize}
\end{block}
\framebreak
\begin{examples}
    \begin{lstlisting}
def line_separator():
    print('')

print("First Line")
line_separator()
print("Second Line")
line_separator()
print("Third Line")
line_separator()
print("Fourth Line")
\end{lstlisting}
\end{examples}
\begin{itemize}
    \item If we want to change the line separator to a dashed line we only need to change a single line of code
\end{itemize}
\begin{lstlisting}
def line_separator():
print('------------------------------')
\end{lstlisting}


\framebreak

\begin{examples}
    \begin{itemize}
        \item If the line seperator should output two lines we can define a new function that calls the \lstinline!line_separator()! function twice
    \end{itemize}
    \begin{lstlisting}
def two_lines():
    line_separator()
    line_separator()

print ("First Line")
two_lines()
print("Second Line")       
    \end{lstlisting}
\end{examples}

\framebreak
\begin{block}{Parameters and arguments}
    \begin{itemize}
        \item Arguments are passed when calling a function
        \item Value of arguments is assigned to parameters
    \end{itemize}
    \begin{lstlisting}
def print_sum(number_1, number_2):
    result = number_1 + number_2
    print(result)

print_sum(1,3)
print_sum(10,5)
    \end{lstlisting}
\end{block}
\begin{block}{Parameters and arguments}
    \framebreak
    \begin{itemize}
        \item Variables are valid within a scope
        \item Variables that are defined in a function can only be seen inside that function
        \item Scope can be identified by indentation
    \end{itemize}
    \begin{lstlisting}
def concatenation(param1, param2):
    concat = part1 + part2
    print(concat)

concatenation("Hello", "World")
print(concat) # NameError: name 'concat' is not defined
    \end{lstlisting}
\end{block}
\begin{block}{Conclusion}
    \begin{itemize}
        \item A function can be called multiple times
        \item If some code can be reused, put it in a function so you need to write less
        \begin{itemize}
            \item Higher factorization
            \item Less redundancy
            \item Better maintenance
        \end{itemize}
        \item Functions can also call other functions
    \end{itemize}
\end{block}
\end{frame}
% subsection functions (end)

\subsection{Naming Conventions \& Debugging} % (fold)
\label{sub:naming_conventions_&_debugging}
\begin{frame}[c,fragile, allowframebreaks]\frametitle{Naming Conventions}
\begin{block}{How to name your functions and variables (PEP8)}
    \begin{itemize}
\item Naming convention is a set of rules for choosing names of functions and variables
\item Every programming language has different naming conventions
\item Python
\begin{itemize}
    \item No spaces in variable and function names
    \item Variable and function names are in lowercase and \_ is used to separate words
\end{itemize}
    \end{itemize}

\begin{lstlisting}
length_in_cm = 15

def say_hello():
    print("Hello")    
\end{lstlisting}
\end{block}
\end{frame}

\begin{frame}[c,allowframebreaks]\frametitle{Debugging}
\begin{block}{Finding and resolving "bugs"}
\begin{itemize}
\item Programming is a complex activity
\item Mistakes happen all the time
\item A mistake made in programming is called a bug
\item The process of finding and resolving bugs is called debugging
\end{itemize}
\end{block}

\begin{block}{Errors}
    \begin{itemize}
        \item Syntax error
        \begin{itemize}
            \item Incorrect syntax of a statement: \lstinline!print(Hello World)! instead of \lstinline!print("Hello World")!
        \end{itemize}
        \item Runtime error
    
    \begin{itemize}
        \item Error that occurs during the execution of a program
        \item e.g. division by 0
    \end{itemize}
        \item Semantic errors
        \begin{itemize}
            \item Program does not deliver correct results
            \item No error messages (code is syntactically correct)
            \item Fixing semantic errors can be extremely complicated (good software design is important)
        \end{itemize}
    \end{itemize}
\end{block}
\framebreak
\begin{block}{Techniques}
\begin{itemize}
        \item Reading code
        \item Print variables with \lstinline!print()! to examine values (a poor man's debugger)
        \item Go through the program step by step -> \textbf{Debugger}!
    \end{itemize}    
\end{block}
    


\end{frame}
% subsection naming_conventions_&_debugging (end)

\subsection{Conditionals} % (fold)
\label{sub:conditionals}

\begin{frame}[c,fragile, allowframebreaks]\frametitle{Conditionals}
\begin{itemize}
\item Boolean algebra is a part of mathematics
\item Often used in programming
\item A boolean expression is either true or false
\end{itemize}

\begin{lstlisting}
5 == 5 # --> True
5 == 6 # --> False
6 > 4 # --> True
5 >= 8 # --> False
\end{lstlisting}

\framebreak
{
    \footnotesize

\begin{examples}

    \begin{block}{\color{mLightBrown}if}

        \begin{itemize}
            \item The expression if defines a condition
            \item If the condition is true, subsequent statements will be executed
            \item If the condition is false, subsequent statements will not be executed
            \item There has to be at least one statement after the condition
        \end{itemize}

    \end{block}
\end{examples}

\begin{lstlisting}
x = 10
if x > 0:
    print(str(x) + ' is positive')
if True:
    # This statement will always be executed
    print('Yes')
if False:
    # This statement will never be executed
    print('No')
\end{lstlisting}
}
\framebreak

    \begin{block}{\color{mLightBrown}else}
        \begin{itemize}
        \item Expression else is executed if the if condition is false
        \item Can only be used in combination with an if expression            
        \end{itemize}
    \end{block}

\begin{lstlisting}
if x == 0:
    print(str(x) + ' is zero')
else:
    print(str(x) + ' is not zero')
\end{lstlisting}

\framebreak

\begin{examples}
    \begin{block}{\color{mLightBrown}\%-operator (remainder after division)}
    {}
    \end{block}
\end{examples}
\begin{lstlisting}
def print_parity(x):
    if x % 2 == 0:
        print(str(x) + ' is even')
    else:
        print(str(x) + ' is odd')

print_parity(2)
print_parity(3)
\end{lstlisting}

\framebreak

\begin{block}{Chained conditionals}
    \begin{itemize}
        \item \lb{elif} is used to combine multiple conditions
        \item The \lb{else} expression is executed when neither \lb{if} nor any of the \lb{elif}s is true.
        \item Any number of \lb{elif} expressions can be used but only one \lb{if} and one \lb{else}
    \end{itemize}
\end{block}


\framebreak

\begin{examples}
    \begin{lstlisting}
if x < y:
    print(str(x) + ' is less than ' + str(y))
elif x > y:
    print(str(x) + ' is greater than ' + str(y))
else:
    print(str(x) + ' and ' + str(y) + ' are equal')
\end{lstlisting}
\begin{lstlisting}
# Python 3
answer = input('Do you like Python?')
# Python 2.7
# answer = raw_input('Do you like Python?')
if answer == 'yes':
    print('That is great!')
else:
    print('That is disappointing!')    
\end{lstlisting}    
\end{examples}
\end{frame}

{
\setbeamercolor{background canvas}{bg=mLightBlue}
\setbeamercolor{frametitle}{fg=white,bg=mLightBlue}
\setbeamercolor{normal text}{fg=white}
\usebeamercolor[fg]{normal text}
\bfseries
\begin{frame}[c, fragile, allowframebreaks]\frametitle{Exercise 1}
    
\begin{itemize}
    \item Write a function \texttt{compare(x,y)} that
    \begin{itemize}
        \item prints 1 if $x > y$
        \item prints 0 if $x == y$
        \item prints -1 if $x < y$
    \end{itemize}
    \item Use \texttt{input()} to receive user input
\end{itemize}

\framebreak

\begin{itemize}
    \item Attention: \texttt{input()} stores the input as a string (not as a number)
    \item If the input is supposed to be a number (int, float) you need to convert it
\end{itemize}

{
\mdseries
\setbeamercolor{normal text}{fg=black}
\usebeamercolor[fg]{normal text}
\begin{lstlisting}
first_number = input('Please enter a first number ')
first_number = int(first_number)
second_number = input('Please enter a first number ')
second_number = int(second_number)
result = first_number + second_number
print(str(result))
\end{lstlisting}
}

\end{frame}
}

\begin{frame}[c, fragile, allowframebreaks]\frametitle{Conditionals}
    \begin{block}{Nested conditionals}
        \begin{itemize}
            \item Conditionals can be nested
        \end{itemize}
\begin{lstlisting}
if x > 0:
    if x < 10:
        print('x is a positive single digit')
\end{lstlisting}
    \end{block}
    \begin{block}{\color{mLightBrown}and}
        \begin{itemize}
            \item Deep nesting can be difficult to read
            \item Use \lb{and} to combine conditionals
        \end{itemize}
    \end{block}
\begin{lstlisting}
if x > 0:
    if x < 10:
        print('x is a positive single digit')
# is the same as
if x > 0 and x < 10:
    print('x is a positive single digit')
\end{lstlisting}


\bblock{\lb{or}}{

    \begin{itemize}
        \item  At least one statement must be true for the condition to be true
    \end{itemize}
}

\begin{lstlisting}
if x > 0 or x < 0:
    print("x is not zero")
\end{lstlisting}

\bblock{\lb{not}}{
    \begin{itemize}
        \item Negation, inverts the boolean.
        \item \lb{not} True -> becomes False
        \item \lb{not} False -> becomes True
    \end{itemize}
}

\begin{lstlisting}
if not (y == 0):
    print(x/y)
else:
    print("Cannot divide by zero")
\end{lstlisting}

\begin{table}
\begin{tabular}{p{1cm}*{3}{p{2cm}}}
\textbf{X} & \textbf{Y} & \textbf{X and Y} & \textbf{X or Y} \\
\toprule
False & False& False& False\\
False& True& False& True\\
True& False &False& True\\
True& True& True& True
\end{tabular}
    
\end{table}


\end{frame}
% subsection conditionals (end)
\subsection{Functions with Return Values} % (fold)
\label{sub:functions_with_return_values}
\begin{frame}[c, allowframebreaks, fragile]\frametitle{Functions with return value}
\begin{itemize}
    \item Some functions will return a value
\end{itemize}

\begin{lstlisting}
# Python 3
answer = input('Do you like Python?')

# Python 2.7
# answer = raw_input('Do you like Python?')
\end{lstlisting}

\begin{itemize}
    \item Our previously defined functions have never returned anything, but only
printed something out
\end{itemize}
\framebreak
\bblock{\lb{return}}{
    \begin{itemize}
        \item Functions that return a value use the \lb{return} keyword
    \end{itemize}
}

\begin{lstlisting}
import math
def area(radius):
    result = math.pi * radius ** 2
    return result

print(area(10))
my_circle_area = area(8)
\end{lstlisting}

\begin{itemize}
    \item Functions can return any valid data type
\end{itemize}

\framebreak

\bblock{\lb{Boolean return values}}{
    \begin{itemize}
        \item The functions can return a boolean vlaue (True, False)
        \item The function name should be formulated as a yes/no question
    \end{itemize}
}

\begin{lstlisting}
def is_divisible(x, y):
    if x % y == 0:
        return True
    else:
        return False
\end{lstlisting}

\framebreak

\bblock{\lb{Boolean return values}}{
    \begin{itemize}
        \item The return value can be used in a condition
    \end{itemize}
}

\begin{lstlisting}
if is_divisible(x, y):
    print(str(x) + ' is divisible by ' + str(y))
else:
    print(str(x) + ' is not divisible by ' + str(y))  
\end{lstlisting}

\end{frame}

{
\setbeamercolor{background canvas}{bg=mLightBlue}
\setbeamercolor{frametitle}{fg=white,bg=mLightBlue}
\setbeamercolor{normal text}{fg=white}
\usebeamercolor[fg]{normal text}
\bfseries
\begin{frame}[c, fragile]\frametitle{Exercise 2}
    
\begin{itemize}
    \item Write a function called \texttt{distance(x1, y1, x2, y2)} which computes the distance between point 1 $(x1, y1)$ and point 2 $(x2, y2)$
    \item Note:
        \begin{itemize}
            \item \( \text{distance} = \sqrt{\left(x_2 - x_1\right)^2 + \left(y_2 - y_1\right)^2} \)
            \item $x^2$ is represented by \texttt{x**2} in Python
            \item The root of x is computed with math.sqrt(x)
            \item Use the import math statement at the beginning of the file
        \end{itemize}
\end{itemize}
\end{frame}

\begin{frame}[c, fragile]\frametitle{Exercise 3}
\begin{itemize}
    \item Write a function \texttt{volume\_from\_radius(radius)}, which calculates the
volume of a sphere
    \item Note:
    \begin{itemize}
        \item \( \text{volume} = \frac{4\pi}{3}\cdot r^3\)
        \item Pi is math.pi
    \end{itemize}
\end{itemize}
    
\end{frame}


\begin{frame}[c, fragile]\frametitle{Exercise 4}
\begin{itemize}
    \item Write a function \texttt{volume\_from\_points(x1, y1, x2, y2)}
    \item  This function calculates the volume of a sphere whose radius is the distance between the points $(x1, y1)$ and $(x2, y2)$
    \item  Tip: Use the implemented methods from the previous exercises
\end{itemize}
    
\end{frame}

\begin{frame}[t, fragile]\frametitle{Exercise 5}
\begin{itemize}
    \item Write a function \texttt{is\_between(x, y, z)} which returns True if x <= y <=
z and False otherwise
\end{itemize}
    
\end{frame}
}

% subsection functions_with_return_values (end)

\subsection{Lists} % (fold)
\label{sub:lists}

\begin{frame}[c, fragile, allowframebreaks]\frametitle{Lists}

    \begin{itemize}
        \item Lists are a data type
        \item Lists are used in most programming languages (arrays)
        \item Lists are a set of values
    \end{itemize}

\begin{lstlisting}
list_a = [1, 2, 4]
list_b = ['Monty', 'Python']
\end{lstlisting}

\framebreak

\bblock{Creating lists}{
    \begin{itemize}
        \item The easiest way to create a list is using []
    \end{itemize}
}

\begin{lstlisting}
numbers = [10, 12, 14, 19]
words = ['spam', 'bungee', 'swallow']
\end{lstlisting}

\begin{itemize}
    \item Data types can be mixed
\end{itemize}   

\begin{lstlisting}
my_list = ['music', 2000, 3.5, True]
\end{lstlisting}

\framebreak

\bblock{Creating lists}{
    \begin{itemize}
        \item Since numbers are often stored in a list, there is a special method for doing so
        \item With only one argument, range returns a number series starting at 0
    \end{itemize}
}

\begin{lstlisting}
list(range(4))
# returns [0, 1, 2, 3]
\end{lstlisting}

\begin{itemize}
    \item When using two arguments it is possible to define the start and end of the range $\left[\text{start}, \text{end}\right)$ (end is not included in the list)
\end{itemize}

\begin{lstlisting}
list(range(1,5))
# returns [1, 2, 3, 4]
\end{lstlisting}

\framebreak

\bblock{Creating lists}{
    \begin{itemize}
        \item The step size can be defined with a third argument
    \end{itemize}
}

\begin{lstlisting}
list(range(1, 10, 2))
# return [1, 3, 5, 7, 9]
\end{lstlisting}

\begin{itemize}
    \item An empty list can also be created
\end{itemize}

\begin{lstlisting}
empty_list = []
\end{lstlisting}
\begin{itemize}
    \item This is often done when the values to be inserted in the list are not yet known.
\end{itemize}

\framebreak

\bblock{Creating lists}{
    \begin{itemize}
        \item Accessing elements can be done with the [] operator
    \end{itemize}
}

\begin{lstlisting}
names = ['Anna', 'Tom', 'Ralph', 'Peter']
print(names[1])
# prints Tom
\end{lstlisting}

\begin{alertblock}{Important}
Array indices start at 0!
\end{alertblock}

\begin{table}
\begin{tabular}{c|c|c|c}
0 & 1 & 2 & 3\\
\midrule
Anna & Tom & Ralph & Peter
\end{tabular}    
\end{table}

\framebreak

\bblock{Accessing lists}{
    \begin{itemize}
        \item A negative index is used to access the list from the end
    \end{itemize}
}

\begin{lstlisting}
names = ['Anna', 'Tom', 'Ralph', 'Peter']
print(names[-1])
# prints Peter
\end{lstlisting}

\framebreak

\bblock{Length}{
    \begin{itemize}
        \item The number of elementsi n a list can be obtained using the \texttt{len()} function
    \end{itemize}
}

\begin{lstlisting}
names = ['Anna', 'Tom', 'Ralph', 'Peter']
print(len(names))
# prints 4    
\end{lstlisting}

\bblock{Out of range}{
    \begin{itemize}
        \item If there is no item in the list at the desired index, Python will print an error
message
    \end{itemize}
}

\begin{lstlisting}
names = ['Anna', 'Tom', 'Ralph', 'Peter']
nNames = len(names)
print(names[nNames])
# IndexError: list index out of range
\end{lstlisting}

\framebreak

\bblock{Changing elements in a list}{
    \begin{itemize}
        \item An element can be changed using [INDEX]
    \end{itemize}
}

\begin{lstlisting}
names = ['Anna', 'Tom', 'Ralph', 'Peter']
names[0] = 'Alice'
# ['Alice', 'Tom', 'Ralph', 'Peter']
\end{lstlisting}

\framebreak

\bblock{Adding elements}
{
    \begin{itemize}
        \item The \texttt{append()} method can be used t o add an element at the end of the list
    \end{itemize}
}

\begin{lstlisting}
numbers = range(5)
# [0, 1, 2, 3, 4]
numbers.append(5)
# [0, 1, 2, 3, 4, 5]  
\end{lstlisting}

\framebreak

\bblock{Concatenate lists}{
    \begin{itemize}
        \item The + operator can be used to join lists
    \end{itemize}
}

\begin{lstlisting}
a = [1, 2, 3]
b = [4, 5, 6]
c = a + b
# [1, 2, 3, 4, 5, 6]
\end{lstlisting}

\framebreak

\bblock{Slices}{
    \begin{itemize}
        \item Lists can be cut into slices
        \item The operator [n:m] returns a list of the elements that start at index n and stop before m
    \end{itemize}
}

\begin{lstlisting}
my_list = ['a', 'b', 'c', 'd', 'e', 'f']
my_list[1:3]
# ['b', 'c']   
\end{lstlisting}

\framebreak

\bblock{Slices}{
    \begin{itemize}
        \item If the first index is empty, the slice starts at the beginning
    \end{itemize}
}

\begin{lstlisting}
my_list = ['a', 'b', 'c', 'd', 'e', 'f']
my_list[:4]
# ['a', 'b', 'c', 'd']   
\end{lstlisting}

\begin{itemize}
    \item If the second index is empty, the slice will include elements until the end of the list
\end{itemize}

\begin{lstlisting}
my_list = ['a', 'b', 'c', 'd', 'e', 'f']
my_list[3:]
# ['d', 'e', 'f']   
\end{lstlisting}

\framebreak

\bblock{Deleting elements}{
    \begin{itemize}
        \item The \texttt{del()} method deletes items from the list
    \end{itemize}
}

\begin{lstlisting}
list_a = ['one', 'two', 'three']
del(list_a[1])
# ['one', 'three']
list_b = ['a', 'b', 'c', 'd', 'e', 'f']
del(list_b[1:5])
# ['a', 'f']
\end{lstlisting}

\end{frame}

\begin{frame}[c, fragile, allowframebreaks]\frametitle{Tuples}

\bblock{Tuples is an immutable sequence data type}{
    \begin{itemize}
        \item It is not possible to assign to the individual items of a tuple, however it is possible to create tuples which contain mutable objects, such as lists.
        \item Tuples are declared using () instead of []
    \end{itemize}
}

\begin{lstlisting}
tuple = ('a', 'b', 'c', 'd', 'e')
\end{lstlisting}

\begin{itemize}
    \item Tuples containing only one element must have a comma at the end of the definition
\end{itemize}
\begin{lstlisting}
tuple = ('a', )
\end{lstlisting}

\end{frame}

\begin{frame}[c, fragile, allowframebreaks]{Strings}

    \bblock{Strings are immutable}{
    \begin{itemize}
        \item Unlike lists, strings cannot be changed
        \item Operations on strings always return a modified copy of the string
        \item The original string remains unchanged
    \end{itemize}
}

\begin{lstlisting}
greeting = 'Hello, world!'
greeting[0] = 'J'
# TypeError: 'str' object does not support item assignment
\end{lstlisting}



\end{frame}


% subsection lists (end)

\subsection{Iteration} % (fold)
\label{sub:iteration}

\begin{frame}[c, fragile, allowframebreaks]\frametitle{Iterations}
    
\begin{itemize}
    \item Iterations are used to repeat statements
    \item There are two expressions for iterations
    \begin{itemize}
        \item \lb{while}
        \item \lb{for}
    \end{itemize}
\end{itemize}

\bblock{\lb{while}}{
    \begin{itemize}
        \item As long as the condition of the while loop is True, the body of the loop gets executed
    \end{itemize}
}

\begin{example}
\begin{lstlisting}
def countdown(n):
    while n > 0:
        print(n)
        n = n - 1
    print('Lift off!')

countdown(10)
\end{lstlisting}    
\end{example}


\bblock{\lb{while}}{
    \begin{itemize}
        \item If the condition is False at the beginning, the body of the loop is never executed
        \item If the variable that is used to check the condition of the while loop does not change, the loop will never terminate -> ininite loop 
        \item Whether a while loop terminates can be hard to determine
    \end{itemize}
}

\begin{lstlisting}
def sequence(n):
    while n != 1:
        print(n)
    if n % 2 == 0:
        n = n / 2
    else:
        n = n * 3 + 1  
\end{lstlisting}

\framebreak

\bblock{\lb{while}}{
    \begin{itemize}
        \item A \lb{while} loop can be used to iterate through a list
    \end{itemize}
}

\begin{lstlisting}
names = ['Tom', 'Anna', 'Christopher']
index = 0
while index < len(names):
    name = names[index]
    print(name)
    index = index + 1
\end{lstlisting}
\end{frame}

{
\setbeamercolor{background canvas}{bg=mLightBlue}
\setbeamercolor{frametitle}{fg=white,bg=mLightBlue}
\setbeamercolor{normal text}{fg=white}
\usebeamercolor[fg]{normal text}
\bfseries
\begin{frame}[c, fragile]\frametitle{Exercise 6}
    
\begin{itemize}
    \item Write a function \texttt{calc\_sum(numbers)} which expects a list of numbers as input and returns their sum
    \item The method should be called as follows \texttt{calc\_sum([4,6,7])}
\end{itemize}
\end{frame}

}

\begin{frame}[c, fragile]\frametitle{Iterations}
    
    \bblock{\lb{for}}{
    \begin{itemize}
        \item Since it is often necessary to operate through lists and other data types, there is a special expression for this
    \end{itemize}
    }

    \begin{lstlisting}
for element in element_list:
    print(element)
    \end{lstlisting}


\end{frame}

{
\setbeamercolor{background canvas}{bg=mLightBlue}
\setbeamercolor{frametitle}{fg=white,bg=mLightBlue}
\setbeamercolor{normal text}{fg=white}
\usebeamercolor[fg]{normal text}
\bfseries
\begin{frame}[c, fragile]\frametitle{Exercise 7}
    
\begin{itemize}
    \item Write a function \texttt{print\_reverse(text)} which expects a string as an argument and prints every character of the string in reverse order
    \item Use a while loop to do this
\end{itemize}
\end{frame}

\begin{frame}[c, fragile]\frametitle{Exercise 8}
\begin{itemize}
\item Write a function \texttt{count\_words(words, min\_word\_length)} that counts the number of words in a list that are at least as long as the specified word length
\item Use a for loop to do this
\item Example:
\end{itemize}

{
\mdseries
\setbeamercolor{normal text}{fg=black}
\usebeamercolor[fg]{normal text}
\begin{lstlisting}
words = ['Emanuel', 'John', 'Ale']
count_words(words, 4)
# 2
\end{lstlisting}
}

\end{frame}

}

% subsection iteration (end)

\subsection{Dictionaries} % (fold)
\label{sub:dictionaries}

\begin{frame}[c, allowframebreaks, fragile]\frametitle{Dictionaries}
    
\bblock{Key-Value pair}{
    \begin{itemize}
        \item Dictionaries are very similar to lists but have a key and value for each entry
        \item The entries of a dictionary are not sorted
    \end{itemize}
}

\framebreak

\bblock{Creating dictionaries}{
    \begin{itemize}
        \item Dictionaries are created using \{\}
    \end{itemize}
}

\begin{lstlisting}
eng2sp = {}
eng2sp['one'] = 'uno'
eng2sp['two'] = 'dos'
\end{lstlisting}

\begin{itemize}
    \item Values can be added directly
\end{itemize}

\begin{lstlisting}
inventory = {
'apples': 430,
'bananas': 312,
}
\end{lstlisting}

\framebreak

\bblock{Accessing entries}{
    \begin{itemize}
        \item Values can be accessed directly using dictionary['key']
    \end{itemize}
}

\begin{lstlisting}
inventory = {
'apples': 430,
'bananas': 312,
}
print(inventory['apples'])
# 430
\end{lstlisting}

\framebreak

\bblock{Assigning and modifying values}{
    \begin{itemize}
        \item The key is assigned a value
        \item If the key already exists the existing value is overwritten
    \end{itemize}
}

\begin{lstlisting}
inventory = {
'apples': 430,
'bananas': 312,
}
inventory['oranges'] = 530
inventory['bananas'] = 250
print(inventory['bananas'])
# 250
\end{lstlisting}

\framebreak

\bblock{Deleting entries}{
    \begin{itemize}
        \item Key-Value pairs can be delted using the \texttt{del()} function
    \end{itemize}
}

\begin{lstlisting}
inventory = {
'apples': 430,
'bananas': 312,
}
del(inventory['bananas'])
\end{lstlisting}

\framebreak

\bblock{Number of entries}{
    \begin{itemize}
        \item The \texttt{len()} function returns the number of entries
    \end{itemize}
}

\begin{lstlisting}
inventory = {
    'apples': 430,
    'bananas': 312,
}
len(inventory)
# 2
\end{lstlisting}

\framebreak

\bblock{Checking if an entry exists}{
    \begin{itemize}
        \item The \lb{in} keyword can be used to check if a key exists in a dictionary
    \end{itemize}
}

\begin{lstlisting}
inventory = {
    'apples': 430,
    'bananas': 312,
}
if 'apples' in inventory:
    inventory['apples'] += 100
else:
    inventory['apples'] = 100
\end{lstlisting}

\framebreak

\bblock{Iterating over entries}{
    \begin{itemize}
        \item The \texttt{items()} function combined with the \lb{for} statement can be used to iterate through every key-value pair
    \end{itemize}
}

\begin{lstlisting}
for (my_key, my_value) in my_dict.items():
    print(my_key + ' : ' + my_value)
\end{lstlisting}

\end{frame}

{
\setbeamercolor{background canvas}{bg=mLightBlue}
\setbeamercolor{frametitle}{fg=white,bg=mLightBlue}
\setbeamercolor{normal text}{fg=white}
\usebeamercolor[fg]{normal text}
\bfseries
\begin{frame}[c, fragile, allowframebreaks]\frametitle{Exercise 9}
    
\begin{itemize}
    \item Write a function \texttt{calculate\_mark(points, max\_points)} which returns a grade in the Swiss grading scale
    \[ \text{mark} = \frac{\text{points}\cdot 5}{\text{max points}} + 1 \]
    \item The function rounds the grade to the nearest 0.5
    \begin{itemize}
        \item 5.66666 -> 5.5
        \item 5.75 -> 6
    \end{itemize}
    \item The function should accept strings as arguments
    \item Arguments should be therefore converted to floats
\end{itemize}

\framebreak

\begin{itemize}
    \item Write a function that asks for points and max\_points as long as the user does not enter "exit"
    \item The grade should be printed after each run
\end{itemize}

{
\mdseries
\setbeamercolor{normal text}{fg=black}
\usebeamercolor[fg]{normal text}
\begin{lstlisting}
while True:
    # input points (use input)
    if points == 'exit':
        break
    # input max_points
    # call calculate_mark function
    # print result
\end{lstlisting}
}
\framebreak

\begin{itemize}
    \item Change your code that it additionally asks for a name
    \item A dictionary should now store the grade of each name
    \begin{itemize}
        \item The name is a key, the grade the value
    \end{itemize}
    \item As soon as the user enters "exit" the program should print the grades of all names before it exits
\end{itemize}

\framebreak

\begin{itemize}
    \item Change your code in such a way that for each name it additionally outputs if the user has passed or failed
    \item Mark >= 4 -> passed
    \item Mark < 4 -> failed
\end{itemize}

\framebreak

\begin{itemize}
    \item Change your in such a way that the application outputs the average grade before it exits
\end{itemize}


\end{frame}

\begin{frame}[c, fragile]\frametitle{Exercise 10}
    
\begin{itemize}
\item  Write an application that generates a random number between 1
and 100
\item  import random
\item  random.randrange(min, max)
\item  The user makes a guess and enters a number. If the number is incorrect, the program outputs whether the entered number was too small or too large and allows the user to guess again.
\item  The application quits when the correct number is guessed
\item  The application should output how many user attempts have been
made before it quits
\end{itemize}

\end{frame}

\begin{frame}[c, fragile]\frametitle{Exercise 11}
    
    \begin{itemize}
        \item Implement the opposite of Task 10 so that the user thinks of a number and makes the computer guess
        \item The user provides feedback on whether the number is too high, too small, or correct
        \item $<$ (too low)
        \item $>$ (too high)
        \item = (correct)
        \item How many steps does the computer need?
    \end{itemize}
\end{frame}

}

% subsection dictionaries (end)
% section fundamental_concepts (end)

\section{Persistence} % (fold)
\label{sec:persistence}

\begin{frame}[c, fragile]\frametitle{Persistence}
\begin{itemize}
    \item So far no data has been saved in any of our examples
    \item All data was deleted from the memory as soon as our examples quit
    \item There are several ways to permanently store data on the hard disk
    \begin{itemize}
        \item Database
        \item Simple text files
    \end{itemize}
\end{itemize}
\end{frame}


\begin{frame}[c, fragile, allowframebreaks]\frametitle{Files}

\bblock{Common procedure}{
    \begin{itemize}
        \item Open file
        \item Do something with the file
        \item Close the file
    \end{itemize}
}

\begin{lstlisting}
file = open('my_file.txt', 'modus')
# do some stuff
file.close()
\end{lstlisting}

\framebreak

\bblock{Different modes}{
    \begin{itemize}
        \item The mode defines how the content of the file should be treated
        \item Modes
        \item'r': read only
        \item'w': write only
        \item'r+': read and write
        \item'a': append
    \end{itemize}
}

\begin{lstlisting}
# open a file in read/write mode
file = open('my_file.txt', 'r')
\end{lstlisting}

\framebreak

\bblock{Write}{
    \begin{itemize}
        \item The write() function is used to write something into a file
        \item \lb{'\textbackslash n'} is used to insert a line break
    \end{itemize}
}

\begin{lstlisting}
file = open('my_file.txt', 'a')
file.write('Das ist eine Linie\n')
file.write('Das ist eine neue Linie\n')
file.close()
\end{lstlisting}

\framebreak

\bblock{Read}{
    \begin{itemize}
        \item A \lb{for} loop can be used to read a file line by line
        \item \texttt{line.strip()} removes the trailing \lb{'\textbackslash n'}
    \end{itemize}
}

\begin{lstlisting}
file = open('my_file.txt', 'r')
for line in file:
    line = line.strip()
    print line
file.close()
\end{lstlisting}
\end{frame}

\begin{frame}[c, fragile, allowframebreaks]\frametitle{JSON}

\bblock{Dictionaries/list in JSON}{
    \begin{itemize}
        \item file.write() only accepts strings as arguments
        \item If complex structures such as dictionaries or lists should be stored in a file, it’s necessary the convert these structures into strings first
        \item An example of a standard used for this purpose is JSON (Javascript Object Notation)
    \end{itemize}
}

\begin{lstlisting}
import json
my_dict = {'one': 'uno', 'two': 'dos'}
my_dict_as_string = json.dumps(my_dict)
print(my_dict_as_string)    
\end{lstlisting}

\framebreak

\bblock{Convert JSON to dictionaries/lists}{
    \begin{itemize}
        \item Example of a string in JSON that is converted into a dictionary
    \end{itemize}
}

\begin{lstlisting}
import json
my_dict_as_string = '{"two": "dos", "one": "uno"}'
my_dict = json.loads(my_dict_as_string)
print(my_dict)
\end{lstlisting}

\end{frame}

{
\setbeamercolor{background canvas}{bg=mLightBlue}
\setbeamercolor{frametitle}{fg=white,bg=mLightBlue}
\setbeamercolor{normal text}{fg=white}
\usebeamercolor[fg]{normal text}
\bfseries
\begin{frame}[c, fragile, allowframebreaks]\frametitle{Exercise 12}
    
\begin{itemize}
\item Write an application which repeatedly asks for a name and phone
number until the user enters ``exit''
\item Each name/telephone number pair should be stored as an entry in a dictionary
\begin{itemize}
    \item The names are the keys of the dictionary
    \item The telephone numbers are the values of the dictionary
\end{itemize}
\item As soon as the user enters ``exit'', create a JSON string of the dictionary using the json.dumps() function and store the string in a file called address\_book.txt
\end{itemize}

\framebreak

\begin{itemize}
\item Extend your application so that it reads the address\_book.txt file when it starts
\item Convert the JSON text into a dictionary again
\end{itemize}

{
\mdseries
\setbeamercolor{normal text}{fg=black}
\usebeamercolor[fg]{normal text}
\begin{lstlisting}
import json
address_book_file = open('address_book.txt', 'r')
address_book_dict = json.load(address_book_file)
\end{lstlisting}
}

\begin{itemize}
\item Ask the user if he wants to add more names or not
\item Let the user search for names in the dictionary and print out the according phone number
\end{itemize}

\end{frame}
}
%section persistence (end)

\begin{frame}[c]\frametitle{Additional Resources}
\begin{itemize}
    \item How to Think Like a Computer Scientist from Allen Downey, Jeffrey Elkner, and Chris Meyers
    \item Learning with Python: Interactive Edition 2.0
    \begin{itemize}
        \item http://interactivepython.org/courselib/static/thinkcspy/index.html
    \end{itemize}

    \item Official Python Documentation
    \begin{itemize}
        \item http://www.python.org/doc/
    \end{itemize}

\item Project Euler: Mathematical problems that can be solved
programmatically
\begin{itemize}
    \item http://projecteuler.net/
\end{itemize}

\item Platforms to prepare for coding interviews
\begin{itemize}
    \item https://leetcode.com/
    \item https://www.interviewbit.com/ 
\end{itemize}

\end{itemize}

\end{frame}

\end{document}
