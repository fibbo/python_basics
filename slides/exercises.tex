%! TEX program = lualatex
\documentclass[10pt, a4paper]{beamer} %handout fuer eine gedruckte version der slides
\usetheme{metropolis}

\usepackage{url}
\usepackage{polyglossia}
\setmainlanguage{english}

\usepackage{listings,xcolor}
\usepackage{graphicx}
\usepackage{nameref}
\usepackage{mathtools}
\usepackage{tikz}
\usetikzlibrary{calc}

\usepackage{booktabs}

\makeatletter
\newcommand*{\currentname}{\@currentlabelname}
\makeatother

\usepackage{fontspec}
\setmainfont{UbuntuMono}
\newfontfamily\Bera{Dank Mono}[Scale=0.85]

\definecolor{mDarkTeal}{HTML}{23373b}
\definecolor{mLightBrown}{HTML}{EB811B}
\definecolor{mDarkBlue}{HTML}{0F3470}
\definecolor{mDarkGrey}{HTML}{999999}
\definecolor{mLighterGrey}{HTML}{CCCCCC}
\definecolor{mLightGrey}{rgb}{0.95,0.95,0.95}
\definecolor{mLightBlue}{HTML}{212751}

\newcommand{\lb}[1]{{\color{mLightBrown}#1}}

\newcommand{\bblock}[3]{
        \begin{block}{#1}
        #2
    \end{block}
}

\setbeamercolor{block title}{fg=mDarkTeal}

\lstset{language=Python,
    backgroundcolor=\color{mLightGrey},
    frame=single,
    rulecolor=\color{mLightGrey}, % make frame "invisible"
    basicstyle=\Bera\footnotesize,
    keywordstyle=\bfseries\color{mDarkBlue},
    commentstyle=\color{mDarkGrey}\itshape,
    captionpos=t,
    emphstyle=\ttb\color{mDarkBlue},    % Custom highlighting style
    stringstyle=\color{mLightBrown},
    tabsize=2,
    numberbychapter=false,
    showstringspaces=false,
    breaklines=true,
    morekeywords={__init__, and, assert, break, class, continue, def, del, elif, else, except, exec, finally, for, from, global, if, import, in, is, lambda, not, or, pass, print, raise, return, try, while, yield}
}
%
\makeatletter
\setbeamertemplate{footline}{%
\leavevmode
\vbox{\begin{beamercolorbox}[dp=1.25ex,ht=2.75ex]{fg=black}%
  \hspace*{1em}\insertsectionhead%
  \ifx\insertsubsectionhead\@empty\relax\else\hspace*{\fill}\insertsubsectionhead\hspace*{2ex}\fi
  \end{beamercolorbox}%
  }%
}
\makeatother

\setbeamertemplate{section in toc}{%
  {\color{mDarkTeal}\rule[0.3ex]{3pt}{3pt}}~\inserttocsection\par}
\setbeamertemplate{subsection in toc}{%
  \hspace{1.2em}{\color{mLightBrown}\rule[0.3ex]{3pt}{3pt}}~\inserttocsubsection\par}


% \usecolortheme{spruce}
\AtBeginSection[]
{
  \begin{frame}
    \frametitle{\currentname}
    \tableofcontents[currentsection]
  \end{frame}
}

\AtBeginSubsection[]
{
  {
  % \setbeamercolor{background canvas}{bg=mLighterGrey}
  \begin{frame}
    \frametitle{\currentname}
    \tableofcontents[currentsubsection]
  \end{frame}
  }
}

\title % (optional, only for long titles)
{Exercises}
\author % (optional, for multiple authors)
{Philipp Gloor\inst{1}}
\institute
{
  \inst{1}%
  University of Zurich
}
\date{}
\subject{Python}
\titlegraphic{\includegraphics[width=0.3\textwidth]{pics/uzh_logo_e_pos}}
\begin{document}
\begin{frame}
	\titlepage
\end{frame}

\begin{frame}{General Rules}
	\begin{itemize}
		\item Always use the function names provided in the exercise
		\item Always create for each exercise a new file!
	\end{itemize}

\end{frame}

{
\setbeamercolor{background canvas}{bg=mLightBlue}
\setbeamercolor{frametitle}{fg=white,bg=mLightBlue}
\setbeamercolor{normal text}{fg=white}
\usebeamercolor[fg]{normal text}
\bfseries
\begin{frame}[c, fragile, allowframebreaks]\frametitle{Exercise 1}

	\begin{itemize}
		\item Write a Python script that
		      \begin{itemize}
			      \item prints 1 if $x > y$
			      \item prints 0 if $x == y$
			      \item prints -1 if $x < y$
		      \end{itemize}
		\item Use \texttt{input()} to receive user input
	\end{itemize}

\end{frame}
\begin{frame}[c, fragile, allowframebreaks]\frametitle{Exercise 1}

	\begin{itemize}
		\item Attention: \texttt{input()} stores the input as a string (not as a number)
		\item If the input is supposed to be a number (int, float) you need to convert it
	\end{itemize}

	{
	\mdseries
	\setbeamercolor{normal text}{fg=black}
	\usebeamercolor[fg]{normal text}
	\begin{lstlisting}
first_number = input('Please enter a first number ')
first_number = int(first_number)
second_number = input('Please enter a first number ')
second_number = int(second_number)
result = first_number + second_number
print(str(result))
\end{lstlisting}
	}

\end{frame}

\begin{frame}[c, fragile]\frametitle{Exercise 2}

	Imagine you have 2 points in a 3d space.
	Each point consists of 3 components: x, y and z.
	One of the points is the center of a sphere, the other is somewhere on the surface of the sphere.

	\begin{center}
	\begin{tikzpicture}[scale=0.8]
		% 3D coordinate system
		\draw[->] (0,0) -- (4,0) node[right] {$x$};
		\draw[->] (0,0) -- (0,3) node[above] {$y$};
		\draw[->] (0,0) -- (-1.5,-1.5) node[below left] {$z$};
		
		% Sphere center point
		\coordinate (center) at (1.5,1.5);
		\fill[red] (center) circle (2pt);
		\node[above right] at (center) {Center $(x_1, y_1, z_1)$};
		
		% Point on surface
		\coordinate (surface) at (3,2.2);
		\fill[blue] (surface) circle (2pt);
		\node[above right] at (surface) {Surface $(x_2, y_2, z_2)$};
		
		% Draw sphere (circle in 2D projection)
		\draw[dashed, gray] (center) circle (1.73);
		
		% Distance line (radius)
		\draw[thick, green] (center) -- (surface);
		\node[midway, above, sloped] at ($(center)!0.5!(surface)$) {radius = distance};
		
		% Grid for reference
		\draw[help lines, very thin] (0,0) grid (4,3);
	\end{tikzpicture}
	\end{center}

	We will try to write code that calculates the volume of a sphere defined by two
	such points
\end{frame}
\begin{frame}[c, fragile]\frametitle{Exercise 2.1}
	\begin{itemize}
		\item Write a function called \texttt{distance(x1, y1, z1, x2, y2, z2)} which computes the distance between point 1 $(x1, y1, z1)$ and point 2 $(x2, y2, z2)$ and returns the result
		\item Note:
		      \begin{itemize}
			      \item \( \text{distance} = \sqrt{\left(x_2 - x_1\right)^2 + \left(y_2 - y_1\right)^2 + \left(z_2 - z_1\right)^2} \)
			      \item $x^2$ is represented by \texttt{x**2} in Python
			      \item The root of x is computed with math.sqrt(x)
			      \item Use the import math statement at the beginning of the file
		      \end{itemize}
	\end{itemize}
\end{frame}

\begin{frame}[c, fragile]\frametitle{Exercise 2.2}
	\begin{itemize}
		\item Write a function \texttt{volume\_from\_radius(radius)}, which calculates the
		      volume of a sphere
		\item Note:
		      \begin{itemize}
			      \item \( \text{volume} = \frac{4\pi}{3}\cdot r^3\)
			      \item Pi is math.pi
		      \end{itemize}
	\end{itemize}

\end{frame}


\begin{frame}[c, fragile]\frametitle{Exercise 2.3}
	Continue in the same file as ex 2.
	\begin{itemize}
		\item Write a function \texttt{volume\_from\_points(x1, y1, z1, x2, y2, z2)}
		\item This function calculates the volume of a sphere whose radius is the distance between the points $(x1, y1, z1)$ and $(x2, y2, z2)$
		\item Tip: Use the implemented methods from the previous exercises (you might want to copy and paste them into a new file)
	\end{itemize}

\end{frame}

\begin{frame}[c, fragile]\frametitle{Exercise 3}
	\begin{itemize}
		\item Write a Python script that reads x, y, and z from `input` and prints True if x <= y <=
		      z and False otherwise
	\end{itemize}

\end{frame}

\begin{frame}[c, fragile]\frametitle{Exercise 4}

	\begin{itemize}
		\item Write a Python script that prints every character of the string in reverse order
		\item Use a while loop to do this
	\end{itemize}
\end{frame}

\begin{frame}[c, fragile]\frametitle{Exercise 5}
	FizzBuzz with a while loop

	\textbf{FizzBuzz} Print "FizzBuzz" if number \% 15 == 0\\
	\textbf{Fizz} Print "Fizz" if number \% 3 == 0\\
	\textbf{Buzz} Print "Buzz" if number \% 5 == 0

	\begin{itemize}
		\item Create a list with numbers from 0 to 100
		\item Write a script that, in a \texttt{while}-loop prints the first element of the list (according to the FizzBuzz rules) and then deletes it from the list
		\item The while loop ends once the list is empty
	\end{itemize}
\end{frame}


\begin{frame}[c, fragile]\frametitle{Exercise 6}
	\begin{itemize}
		\item Write a script that counts the number of words in a list that are at least as long
		      as the specified word length.
		\item Use a for loop to do this
		\item Example:
	\end{itemize}

		{
		\mdseries
		\setbeamercolor{normal text}{fg=black}
		\usebeamercolor[fg]{normal text}
		\begin{lstlisting}
words = ['apple', 'banana', 'cat', 'elephant', 'dog']
min_length = 4
count = 0
# Use a for loop to count words with length >= min_length
# Should print 3 (apple=5, banana=6, elephant=8)
\end{lstlisting}
		}

\end{frame}

\begin{frame}[c, fragile, allowframebreaks]\frametitle{Exercise 7.1}

	\begin{itemize}
		\item Write a function \texttt{calculate\_mark(points, max\_points)} which returns a grade in the Swiss grading scale
		      \[ \text{mark} = \frac{\text{points}\cdot 5}{\text{max points}} + 1 \]
		\item The function rounds the grade to the nearest 0.5
		      \begin{itemize}
			      \item 5.66666 -> 5.5
			      \item 5.75 -> 6
		      \end{itemize}
		\item The function should accept strings as arguments
		\item Arguments should be therefore converted to floats
	\end{itemize}

\end{frame}
\begin{frame}[c, fragile, allowframebreaks]\frametitle{Exercise 7.2}

	\begin{itemize}
		\item Ask for max\_points via input() function
		\item Write code that in a while loop asks for points as long as the user does not enter "exit"
		\item The grade should be printed after each run
	\end{itemize}

	{
	\mdseries
	\setbeamercolor{normal text}{fg=black}
	\usebeamercolor[fg]{normal text}
	\begin{lstlisting}
# Read max_points from input
while True:
    # Read points from input
    if points == 'exit':
        break
    # call calculate_mark function
    # print result
\end{lstlisting}
	}
\end{frame}
\begin{frame}[c, fragile, allowframebreaks]\frametitle{Exercise 7.3}

	\begin{itemize}
		\item Change your code that it additionally asks for a name
		\item A dictionary should now store the grade of each name
		      \begin{itemize}
			      \item The name is a key, the grade the value
		      \end{itemize}
		\item As soon as the user enters "exit" the program should print the grades of all names before it exits
	\end{itemize}

\end{frame}
\begin{frame}[c, fragile, allowframebreaks]\frametitle{Exercise 7.4}

	\begin{itemize}
		\item Change your code in such a way that for each name it additionally outputs if the user has passed or failed
		\item Mark >= 4 -> passed
		\item Mark < 4 -> failed
	\end{itemize}
\end{frame}

\begin{frame}[c, fragile, allowframebreaks]\frametitle{Exercise 7.5}

	\begin{itemize}
		\item Change your in such a way that the application outputs the average grade before it exits
	\end{itemize}


\end{frame}

\begin{frame}[c, fragile]\frametitle{Exercise 8}

	\begin{itemize}
		\item  Write an application that generates a random number between 1
		      and 100
		\item  import random
		\item  random.randrange(min, max)
		\item  The user makes a guess and enters a number. If the number is incorrect, the program outputs whether the entered number was too small or too large and allows the user to guess again.
		\item  The application quits when the correct number is guessed
		\item  The application should output how many user attempts have been
		      made before it quits
	\end{itemize}

\end{frame}

\begin{frame}[c, fragile]\frametitle{Exercise 9}

	\begin{itemize}
		\item Implement the opposite of Task 8 so that the user thinks of a number and makes the computer guess
		\item The user provides feedback on whether the number is too high, too small, or correct
		\item $<$ (too low)
		\item $>$ (too high)
		\item = (correct)
		\item How many steps does the computer need?
	\end{itemize}
\end{frame}


\begin{frame}[c, fragile, allowframebreaks]\frametitle{Exercise 10}

	\begin{itemize}
		\item Write an application which repeatedly asks for a name and phone
		      number until the user enters ``exit''
		\item Each name/telephone number pair should be stored as an entry in a dictionary
		      \begin{itemize}
			      \item The names are the keys of the dictionary
			      \item The telephone numbers are the values of the dictionary
		      \end{itemize}
		\item As soon as the user enters ``exit'', create a JSON string of the dictionary using the json.dumps() function and store the string in a file called address\_book.txt
	\end{itemize}

	\framebreak

	\begin{itemize}
		\item Extend your application so that it reads the address\_book.txt file when it starts
		\item Convert the JSON text into a dictionary again
	\end{itemize}

	{
	\mdseries
	\setbeamercolor{normal text}{fg=black}
	\usebeamercolor[fg]{normal text}
	\begin{lstlisting}
import json
address_book_file = open('address_book.txt', 'r')
address_book_dict = json.load(address_book_file)
\end{lstlisting}
	}

	\begin{itemize}
		\item Ask the user if he wants to add more names or not
		\item Let the user search for names in the dictionary and print out the according phone number
	\end{itemize}

\end{frame}

\begin{frame}[c, fragile, allowframebreaks]\frametitle{Exercise 11}

	\begin{itemize}
		\item Write an application which repeatedly asks for a name and phone
		      number until the user enters ``exit''
		\item Each name/telephone number pair should be stored as an entry in a dictionary
		      \begin{itemize}
			      \item The names are the keys of the dictionary
			      \item The telephone numbers are the values of the dictionary
		      \end{itemize}
		\item As soon as the user enters ``exit'', create a JSON string of the dictionary using the json.dumps() function and store the string in a file called address\_book.txt
	\end{itemize}

	\framebreak

	\begin{itemize}
		\item Extend your application so that it reads the address\_book.txt file when it starts
		\item Convert the JSON text into a dictionary again
	\end{itemize}

	{
	\mdseries
	\setbeamercolor{normal text}{fg=black}
	\usebeamercolor[fg]{normal text}
	\begin{lstlisting}
import json
address_book_file = open('address_book.txt', 'r')
address_book_dict = json.load(address_book_file)
\end{lstlisting}
	}

	\begin{itemize}
		\item Ask the user if he wants to add more names or not
		\item Let the user search for names in the dictionary and print out the according phone number
	\end{itemize}

\end{frame}
}

\end{document}
